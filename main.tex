% !TeX program = xelatex

%___________________________________________________________________
% Load Class

\documentclass[
  fontsize=10pt, 
  aspectratio=169,
  xcolor={dvipsnames}
]{beamer}
% beamer auto loads hyperref, color, xcolor, amsthm

%___________________________________________________________________
% Load Packages

% \usepackage[T1]{fontenc}        % * only if font supports T1 *
% \usepackage[utf8]{inputenc}     % * redundant for modern latex versions *

\usepackage{fontspec}               % required for system fonts
\usepackage{microtype}         % better typography
\usepackage{polyglossia}    % languages

% % \usepackage{enumitem}

% citations
\usepackage[
  backend=biber,              
  style=ieee,                 
]{biblatex}   
\usepackage[
  autostyle=true, 
  german=quotes
]{csquotes}

% better unit display
\usepackage[
  locale=DE,
  mode=match,
  % per-mode-fraction,
  range-units=repeat,
  range-phrase={{~bis~}}
]{siunitx}

\usepackage{amsmath}

% input external data
\usepackage{listings}
\usepackage{graphicx} 
\usepackage{svg}

% improve visualization of data
\usepackage{booktabs}
\usepackage{tabularx}
\usepackage{longtable}
\usepackage{subcaption}	

\usepackage{appendixnumberbeamer}

%___________________________________________________________________
% Dimensions

\newlength{\imagewidth}	

%___________________________________________________________________
% Colors

\definecolor{bluelinks}{rgb}{0.16, 0.32, 0.75}

\newcommand{\commentcolor}{\color{ForestGreen}}
\newcommand{\stringcolor}{\color{Mulberry}}
\newcommand{\keywordcolor}{\color{blue}}

%___________________________________________________________________
% New Makros

\newcommand{\imagesource}[1]{\par\footnotesize\textbf{Quelle:}~#1}   	%   Quellen für Bilder
\newcommand{\imagelegend}[1]{\par\footnotesize\textbf{Legende:}~#1}

%___________________________________________________________________
% Renamings

\renewcommand{\lstlistingname}{Quellcode}						%	Snippet umbenennen
\renewcommand{\lstlistlistingname}{Codeverzeichnis}	

\addto\captionsgerman{%
  \renewcommand{\proofname}{Beweis}
}

\renewcommand{\mkcitation}[1]{#1}

%___________________________________________________________________
% Commands

\usetheme[]{metropolis}     % beamer style

\setdefaultlanguage{german}
\setotherlanguage{english}

\addbibresource{sources.bib}

\graphicspath{{/images}}  

\hypersetup{
%   hidelinks,
  bookmarksnumbered=true,
  breaklinks=true,
  pdfauthor=Jan Hoegen,
  pdftitle=Pytorch NNB
}

%___________________________________________________________________
% Listings

\lstset{%
	frame			=	tb ,							%	horizontale Linie oben&unten
	breaklines		=	true,							%	Zeilenumbruch
	rulecolor		=	\color{black} ,					%	Rahmenfarbe ist schwarz
	keywordstyle	=	\keywordcolor ,
	commentstyle	=	\commentcolor ,
	stringstyle		=	\stringcolor ,
	title			=	\lstname ,						%	Titel ist gleich dem Dateinamen
	basicstyle		=	\footnotesize\ttfamily ,		%	Kleine Schrift und Monospace
	numbers			=	left,							%	Zeilennumber links
	inputencoding	=	utf8,  							% Input encoding
  extendedchars	=	true,  							% Extended ASCII
}

% \lstset{literate=							%	ermöglicht Unicode-Zeichen in Listing!  % * unnötig, erzeugt Fehlerhafte Darstellung *
%   {á}{{\'a}}1 {é}{{\'e}}1 {í}{{\'i}}1 {ó}{{\'o}}1 {ú}{{\'u}}1
%   {Á}{{\'A}}1 {É}{{\'E}}1 {Í}{{\'I}}1 {Ó}{{\'O}}1 {Ú}{{\'U}}1
%   {à}{{\`a}}1 {è}{{\`e}}1 {ì}{{\`i}}1 {ò}{{\`o}}1 {ù}{{\`u}}1
%   {À}{{\`A}}1 {È}{{\'E}}1 {Ì}{{\`I}}1 {Ò}{{\`O}}1 {Ù}{{\`U}}1
%   {ä}{{\"a}}1 {ë}{{\"e}}1 {ï}{{\"i}}1 {ö}{{\"o}}1 {ü}{{\"u}}1
%   {Ä}{{\"A}}1 {Ë}{{\"E}}1 {Ï}{{\"I}}1 {Ö}{{\"O}}1 {Ü}{{\"U}}1
%   {â}{{\^a}}1 {ê}{{\^e}}1 {î}{{\^i}}1 {ô}{{\^o}}1 {û}{{\^u}}1
%   {Â}{{\^A}}1 {Ê}{{\^E}}1 {Î}{{\^I}}1 {Ô}{{\^O}}1 {Û}{{\^U}}1
%   {ã}{{\~a}}1 {?}{{\~e}}1 {i}{{\~i}}1 {õ}{{\~o}}1 {u}{{\~u}}1
%   {Ã}{{\~A}}1 {?}{{\~E}}1 {I}{{\~I}}1 {Õ}{{\~O}}1 {U}{{\~U}}1
%   {œ}{{\oe}}1 {Œ}{{\OE}}1 {æ}{{\ae}}1 {Æ}{{\AE}}1 {ß}{{\ss}}1
%   {u}{{\H{u}}}1 {U}{{\H{U}}}1 {o}{{\H{o}}}1 {O}{{\H{O}}}1
%   {ç}{{\c c}}1 {Ç}{{\c C}}1 {ø}{{\o}}1 {å}{{\r a}}1 {Å}{{\r A}}1
%   {€}{{\euro}}1 {£}{{\pounds}}1 {«}{{\guillemotleft}}1
%   {»}{{\guillemotright}}1 {ñ}{{\~n}}1 {Ñ}{{\~N}}1 {¿}{{?`}}1 {¡}{{!`}}1 
%   {~}{{\textasciitilde}}1 {*}{{\normalfont{*}}}1
% }

%___________________________________________________________________
% Ttile Page

\title{CIFAR10 mit PyTorch klassifizieren}
\subtitle{Neuronale Netze in der Bildverarbeitung}
\date{\today}
\author{Jan Hoegen \and Nico Weber}
\institute{
  Hochschule Karlsruhe\\
  University of Applied Sciences
}
% \setbeamertemplate{frame footer}{My custom footer}

%___________________________________________________________________
% Document

\begin{document}
  \maketitle

  \begin{frame}{Inhaltsverzeichnis}
    \setbeamertemplate{section in toc}[sections numbered]
    \tableofcontents[hideallsubsections]
    % \tableofcontents
  \end{frame}

  \section{Allgemeines zu PyTorch}

%___________________________________________________________________

\begin{frame}{PyTorch im Überblick}
    \begin{columns}[T,onlytextwidth]
        \column{0.5\textwidth}
        \textbf{Was ist PyTorch?}
        \begin{itemize}
            \item Python-Bibliothek für Deep Learning
            \item Stark verbreitet in Forschung und Lehre \cite{bauer}
            \item Unterstützt dynamische Berechnungsgraphen („Define-by-Run“)
        \end{itemize}

        \column{0.5\textwidth}
        \textbf{Warum PyTorch?}
        \begin{itemize}
            \item Einfache und flexible Modellimplementierung
            \item Direkte Nutzung von GPU-Beschleunigung
            \item Große Community, viele Tutorials und Ressourcen
        \end{itemize}
    \end{columns}
\end{frame}

%___________________________________________________________________

\begin{frame}{Grundlegende Konzepte von Pytorch}
    \begin{description}
        \item[Autograd] Berechnet Gradienten automatisch für Backpropagation
        \item[\texttt{nn.Module}] Basis für selbstdefinierte Modelle, enthält vordefinierte Layers
        \item[DataLoader] Einfaches Laden, Batchen und \alert{Parallelisieren} von Datensätzen
        \item[Optimizer] Vorgefertigte Optimierer, z.B. \texttt{Adam}
    \end{description}
\end{frame}

  \section{Eigenes Modell zu CIFAR 10}

%___________________________________________________________________

\begin{frame}{Aufgabenstellung}
    \begin{itemize}
        \item \textbf{Aufgabe:} Klassifikation von CIFAR-10 Bildern
        \item \textbf{Datensatz:}
            \begin{itemize}
                \item 10 Klassen
                \item 60.000 Bilder, Größe $32\times32\times3$
                \item Trainingsset: 49.000 Bilder
                \item Validierungsset: 1.000 Bilder
                \item Testset: 10.000 Bilder
            \end{itemize}
        \item \textbf{Ziel:} Modell in \emph{maximal 10 Epochen} trainieren, um Bilder korrekt zu klassifizieren
    \end{itemize}
\end{frame}



%___________________________________________________________________

\begin{frame}{Architektur des Modells}
    \begin{figure}
        \centering
        \includegraphics[width=\imagewidth, height=\imageheight, keepaspectratio]{reduced_graph.pdf}
    \end{figure}

\end{frame}

%___________________________________________________________________

\begin{frame}{Ergebnisse des besten Modells}

\begin{itemize}
    \item \textbf{Gewählte Hyperparameter:}
        \begin{itemize}
            \item ch1 = 64, ch2 = 64, ch3 = 64
            \item Lernrate = 0.00086
            \item Weight Decay = 3.81e-6
        \end{itemize}
    \item \textbf{Best Validation Accuracy:} \alert{75.70\%}
    
    \item \textbf{Trainings-Accuracy:} \alert{73\%}

    \item \textbf{Test Accuracy:} \alert{75.05\%}
\end{itemize}
\end{frame}

%___________________________________________________________________

\begin{frame}{Accuracy für verschiedene Klassen}
    \begin{figure}
        \centering
        \includegraphics[width=\imagewidth, height=\imageheight, keepaspectratio]{class_accuracy.png}
    \end{figure}
\end{frame}

%___________________________________________________________________

\begin{frame}{Confusion Matrix}
    \begin{figure}
        \centering
        \includegraphics[width=\imagewidth, height=\imageheight, keepaspectratio]{confusion_matrix.png}
    \end{figure}
\end{frame}

%___________________________________________________________________

\begin{frame}{Beispielvorhersagen}
    \begin{figure}
        \centering
        \includegraphics[width=\imagewidth, height=\imageheight, keepaspectratio]{predictions_grid.png}
    \end{figure}
\end{frame}



  \section{Anwendung des Eigenen Modeels}

%  TODO: Name ändern

\begin{frame}
    Zeige CUDA batch size, workers, iterations, epochen, trails, zeitdauer
\end{frame}

\begin{frame}{optimizer AdamW}
    
\end{frame}

%___________________________________________________________________

\begin{frame}{scheduler CosineAnnealingLR}
    
\end{frame}

%___________________________________________________________________

\begin{frame}{Patience early stopping}
    
\end{frame}

%___________________________________________________________________




  \section{Parametertuning mit Bayesian Search}

%___________________________________________________________________
\begin{frame}{Bayesian Search Erklärt}
\begin{itemize}
    \item Bayesian Search ist eine intelligente Methode zur Optimierung von Hyperparametern.
    \item Ziel: Maximierung der Validierungsgenauigkeit $f(\text{Parameter}) \rightarrow \text{Validation Accuracy}$.
    \item Idee:
        \begin{itemize}
            \item Es wird ein Modell erstellt, das die unbekannte Zielfunktion $f$ abschätzt.
            \item Nach jeder Evaluierung wird dieses Modell mit den neuen Ergebnissen aktualisiert.
            \item Eine \textit{Acquisition Function} wählt die nächsten Parameterwerte aus – als Kompromiss zwischen \textbf{Exploration} (neue Bereiche testen) und \textbf{Exploitation} (bestehende gute Bereiche verfeinern).
        \end{itemize}
    \item Vorteil: Findet gute Parameter mit deutlich weniger Versuchen als Random oder Grid Search.
\end{itemize}
\end{frame}

%___________________________________________________________________
\begin{frame}{Animation zu Bayesian Search}
\begin{columns}[T, totalwidth=\textwidth]

    % -------------------------------------------------------------
    % Linke Spalte: Animation
    \begin{column}{0.55\textwidth}
        \centering
        \animategraphics[loop, autoplay, controls, width=\linewidth, height=\imageheight, keepaspectratio]{0.5}{videos/frame_}{000}{006}
    \end{column}

    % -------------------------------------------------------------
    % Rechte Spalte: Caption und Beschreibung
    \begin{column}{0.45\textwidth}
    \captionof{figure}{Bayes'sche Optimierung eines Scores für einen Random-Forest-Klassifizierer}
    \imagesource{\cite{bayes_gif}}
    \imagelegend{
        \scriptsize
        \textit{x-Achse:} Parameter des Random-Forest-Klassifizierers. \textit{Schwarz:} Zielfunktion. \textit{Lila:} Modellierte Funktion mit Unsicherheitsbereich ±1 Standardabweichung. \\ 
        \textit{Expected Improvement:} Erwarteter Zugewinn gegenüber dem aktuellen Bestwert. \\ 
        \textit{Upper Confidence Bound:} Suche vielversprechende, aber unerkundete Bereiche. \\
        \textit{Probability of Improvement:} Wahrscheinlichkeit, dass ein neuer Punkt besser ist als der bisherige Bestwert.
    }
\end{column}

\end{columns}
\end{frame}

%___________________________________________________________________
\begin{frame}{Anwendung auf unser Modell}
\begin{columns}[T,onlytextwidth]
    \begin{column}{0.52\textwidth}
        \begin{itemize}
            \item Anstatt zufällig Parameterkombinationen zu testen (Random Search) oder alle möglichen Kombinationen (Grid Search), wurde \textbf{Bayesian Search} verwendet.
            \item Das Modell der Funktion $\text{Validation Accuracy} = f(\text{Hyperparameter})$ wird kontinuierlich angepasst.
            \item Neue Vorschläge für Hyperparameter werden auf Basis bisheriger Ergebnisse erzeugt.
        \end{itemize}
    \end{column}
    \begin{column}{0.45\textwidth}
        \begin{figure}
            \includegraphics[width=\linewidth, height=\imageheight, keepaspectratio]{param_importances.png}
            \caption{Relative Wichtigkeit der Hyperparameter}
        \end{figure}
    \end{column}
\end{columns}
\end{frame}

%___________________________________________________________________
\begin{frame}{Bayesian Search Ergebnisse}
\begin{figure}
    \centering
    \includegraphics[width=\imagewidth, height=\imageheight, keepaspectratio]{optimization_history.png}
    \caption{Verlauf der Validierungsgenauigkeit über die Trials (je höher, desto besser).}

\end{figure}
\end{frame}


%   \begin{frame}{Blocks}
%   Three different block environments are pre-defined and may be styled with an
%   optional background color.

%   \begin{columns}[T,onlytextwidth]
%     \column{0.5\textwidth}
%       \begin{block}{Default}
%         Block content.
%       \end{block}

%       \begin{alertblock}{Alert}
%         Block content.
%       \end{alertblock}

%       \begin{exampleblock}{Example}
%         Block content.
%       \end{exampleblock}

%     \column{0.5\textwidth}

%       \metroset{block=fill}

%       \begin{block}{Default}
%         Block content.
%       \end{block}

%       \begin{alertblock}{Alert}
%         Block content.
%       \end{alertblock}

%       \begin{exampleblock}{Example}
%         Block content.
%       \end{exampleblock}

%   \end{columns}
% \end{frame}

% \section{Second Section}
%   \begin{frame}{Second Frame}
%       The theme provides sensible defaults to
% \emph{emphasize} text, \alert{accent} parts
% or show \textbf{bold} results.
%   \end{frame}

%   \begin{frame}{References}
%   Some references to showcase [allowframebreaks] \cite{Knuth92,ConcreteMath,Simpson,Er01,greenwade93}
%   \cite{ConcreteMath}
% \end{frame}

% \begin{frame}{Animation}
%   \begin{itemize}[<+- | alert@+>]
%     \item \alert<4>{This is\only<4>{ really} important}
%     \item Now this
%     \item And now this
%   \end{itemize}
% \end{frame}

\begin{frame}[label=conclusion, standout]
  Fragen?
\end{frame}

\appendix

\begin{frame}[allowframebreaks]{Literatur}
  \nocite{*}
  \printbibliography[heading=none]
\end{frame}

\end{document}