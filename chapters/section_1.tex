\section{Allgemeines zu PyTorch}

%___________________________________________________________________

\begin{frame}{PyTorch im Überblick}
    \begin{columns}[T,onlytextwidth]
        \column{0.5\textwidth}
        \textbf{Was ist PyTorch?}
        \begin{itemize}
            \item Python-Bibliothek für Deep Learning
            \item Stark verbreitet in Forschung und Lehre \cite{bauer}
            \item Unterstützt dynamische Berechnungsgraphen („Define-by-Run“)
        \end{itemize}

        \column{0.5\textwidth}
        \textbf{Warum PyTorch?}
        \begin{itemize}
            \item Einfache und flexible Modellimplementierung
            \item Direkte Nutzung von GPU-Beschleunigung
            \item Große Community, viele Tutorials und Ressourcen
        \end{itemize}
    \end{columns}
\end{frame}

%___________________________________________________________________

\begin{frame}{Grundlegende Konzepte von Pytorch}
    \begin{description}
        \item[Autograd] Berechnet Gradienten automatisch für Backpropagation
        \item[\texttt{nn.Module}] Basis für selbstdefinierte Modelle, enthält vordefinierte Layers
        \item[DataLoader] Einfaches Laden, Batchen und \alert{Parallelisieren} von Datensätzen
        \item[Optimizer] Vorgefertigte Optimierer, z.B. \texttt{Adam}
    \end{description}
\end{frame}