\section{Parametertuning mit Bayesian Search}

%___________________________________________________________________
\begin{frame}{Bayesian Search Erklärt}
\begin{itemize}
    \item Bayesian Search ist eine intelligente Methode zur Optimierung von Hyperparametern.
    \item Ziel: Maximierung der Validierungsgenauigkeit $f(\text{Parameter}) \rightarrow \text{Validation Accuracy}$.
    \item Idee:
        \begin{itemize}
            \item Es wird ein Modell erstellt, das die unbekannte Zielfunktion $f$ abschätzt.
            \item Nach jeder Evaluierung wird dieses Modell mit den neuen Ergebnissen aktualisiert.
            \item Eine \textit{Acquisition Function} wählt die nächsten Parameterwerte aus – als Kompromiss zwischen \textbf{Exploration} (neue Bereiche testen) und \textbf{Exploitation} (bestehende gute Bereiche verfeinern).
        \end{itemize}
    \item Vorteil: Findet gute Parameter mit deutlich weniger Versuchen als Random oder Grid Search.
\end{itemize}
\end{frame}

%___________________________________________________________________
\begin{frame}{Animation zu Bayesian Search}
\begin{columns}[T, totalwidth=\textwidth]

    % -------------------------------------------------------------
    % Linke Spalte: Animation
    \begin{column}{0.55\textwidth}
        \centering
        \animategraphics[loop, autoplay, controls, width=\linewidth, height=\imageheight, keepaspectratio]{0.5}{videos/frame_}{000}{006}
    \end{column}

    % -------------------------------------------------------------
    % Rechte Spalte: Caption und Beschreibung
    \begin{column}{0.45\textwidth}
    \captionof{figure}{Bayes'sche Optimierung eines Scores für einen Random-Forest-Klassifizierer}
    \imagesource{\cite{bayes_gif}}
    \imagelegend{
        \scriptsize
        \textit{x-Achse:} Parameter des Random-Forest-Klassifizierers. \textit{Schwarz:} Zielfunktion. \textit{Lila:} Modellierte Funktion mit Unsicherheitsbereich ±1 Standardabweichung. \\ 
        \textit{Expected Improvement:} Erwarteter Zugewinn gegenüber dem aktuellen Bestwert. \\ 
        \textit{Upper Confidence Bound:} Suche vielversprechende, aber unerkundete Bereiche. \\
        \textit{Probability of Improvement:} Wahrscheinlichkeit, dass ein neuer Punkt besser ist als der bisherige Bestwert.
    }
\end{column}

\end{columns}
\end{frame}

%___________________________________________________________________
\begin{frame}{Anwendung auf unser Modell}
\begin{columns}[T,onlytextwidth]
    \begin{column}{0.52\textwidth}
        \begin{itemize}
            \item Anstatt zufällig Parameterkombinationen zu testen (Random Search) oder alle möglichen Kombinationen (Grid Search), wurde \textbf{Bayesian Search} verwendet.
            \item Das Modell der Funktion $\text{Validation Accuracy} = f(\text{Hyperparameter})$ wird kontinuierlich angepasst.
            \item Neue Vorschläge für Hyperparameter werden auf Basis bisheriger Ergebnisse erzeugt.
        \end{itemize}
    \end{column}
    \begin{column}{0.45\textwidth}
        \begin{figure}
            \includegraphics[width=\linewidth, height=\imageheight, keepaspectratio]{param_importances.png}
            \caption{Relative Wichtigkeit der Hyperparameter}
        \end{figure}
    \end{column}
\end{columns}
\end{frame}

%___________________________________________________________________
\begin{frame}{Bayesian Search Ergebnisse}
\begin{figure}
    \centering
    \includegraphics[width=\imagewidth, height=\imageheight, keepaspectratio]{optimization_history.png}
    \caption{Verlauf der Validierungsgenauigkeit über die Trials (je höher, desto besser).}

\end{figure}
\end{frame}
